% ----------------------------------------------------------------------- %
% Arquivo: 7-consideracoes-finais.tex
% ----------------------------------------------------------------------- %

\chapter{Considerações Finais}
\label{c_consideracoes-finais}

A automatização da criação da infraestrutura remota e seu gerenciamento através de código utilizando o Terraform facilitou o desenvolvimento do trabalho. Com o Terraform é possível não apenas adicionar componentes da nuvem pública utilizada, mas também aprimorar sua configuração ou removê-los da infraestrutura. A abstração fornecida pelo gerenciamento dos recursos através de código é interessante não apenas pela automatização, mas também por possibilitar a adoção de um padrão de versionamento de infraestrutura.

A utilização do Kubernetes como orquestrador de contêineres do \textit{cluster} também foi interessante devido a automatização de tarefas oferecida. É possível, por exemplo, configurar a escalabilidade do serviço implantado através de replicações automáticas de \textit{pods} caso a carga de trabalho aumente. Assim, se a carga de trabalho do \textit{pod} WordPress aumentar, é possível que o orquestrador automaticamente crie novos \textit{pods} para distribuir a carga de trabalho. Além disso, caso ocorra a falha de um \textit{pod}, o Kubernetes irá automaticamente reiniciar este \textit{pod}, montando novamente os volumes anteriormente utilizados e gerenciados pelo Rook. O orquestrador possibilita, portanto, que serviços sejam automaticamente reinicializados caso necessário, dispensando assim a intervenção direta do administrador para tratar falhas mais simples de serviços.

O Rook, por sua vez, mostrou mostrou ser uma alternativa interessante para o fornecimento de armazenamento distribuído em uma infraestrutura hiperconvergente de contêineres. A possibilidade de ofertar armazenamento através de blocos, arquivos e objetos é um ponto positivo, possibilitando que o administrador do serviço possa escolher o tipo de armazenamento que melhor se adeque as suas necessidades. Ao utilizar o cliente da \textit{Amazon S3} para acessar e gerenciar o \textit{bucket} criado no Rook foi possível perceber a similaridade entre os serviços, o que também é um ponto positivo já que a \textit{Amazon S3} é um serviço de armazenamento consolidado no mercado. Esta similaridade permite, por exemplo, que dados que são armazenados em um \textit{cluster} Rook sejam posteriormente transferidos para a \textit{Amazon S3}, ou vice-versa.

\section{Trabalhos futuros}

Sugerem-se os seguintes temas para trabalhos futuros:

\begin{itemize}
    \item Estudo de necessidades de armazenamento da rede do IFSC
    \item Proposta de implantação de sistema de armazenamento distribuído baseado em contêineres no IFSC
    \item Proposta de implantação de infraestrutura hiperconvergente no IFSC
\end{itemize}